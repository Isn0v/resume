% !TeX root = ../resume.tex

% --- ВСПОМОГАТЕЛЬНЫЕ КОМАНДЫ ---

% Команда для заголовков секций (WORK EXPERIENCE, SKILLS и т.д.)
\newcommand{\sectiontitle}[1]{%
    \vspace{8pt}
    \par
    \noindent\MakeUppercase{#1}
    \hrule width 1\textwidth height 0.4pt
    \vspace{8pt}
}

% Команда для иконки внешней ссылки
\newcommand{\extlink}{\, \textcolor{graytext}{\faIcon{external-link-alt}}}

% Команда для пунктов в списке достижений
\setlist[itemize,1]{label=\,--, leftmargin=*, topsep=0pt, itemsep=4pt}

% Команда для описания языков
\newcommand{\languageskill}[2]{%
    \par\noindent\hangindent=1em\hangafter=0
    \makebox[8em][l]{#1} % Название языка
    \foreach \i in {1,...,5}{
        \ifnum\i > #2
            \tikz\draw[graytext, fill=white] (0,0) circle (2.5pt);
        \else
            \tikz\fill[maintext] (0,0) circle (2.5pt);
        \fi
        \hspace{2pt}
    }
    \vspace{4pt}
}

% Команда для "таблеток" с интересами
\newcommand{\interesttag}[1]{%
    \tikz[baseline=(X.base)] \node[draw=graytext, rounded corners=3pt, inner sep=4pt, text=graytext] (X) {#1};
}